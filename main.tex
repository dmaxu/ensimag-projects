\documentclass{article}

% Language setting
% Replace `english' with e.g. `spanish' to change the document language
\usepackage[english]{babel}

% Set page size and margins
% Replace `letterpaper' with `a4paper' for UK/EU standard size
\usepackage[letterpaper,top=2cm,bottom=2cm,left=3cm,right=3cm,marginparwidth=1.75cm]{geometry}
\usepackage{longtable} % Ajoutez cette ligne au préambule pour activer longtable

% Useful packages
\usepackage{amsmath}
\usepackage{graphicx}
\usepackage[colorlinks=true, allcolors=blue]{hyperref}
\usepackage{array}
\usepackage{graphicx} % For \resizebox
\setlength{\arrayrulewidth}{0.6mm}
\renewcommand{\arraystretch}{1.5}

\title{Documentation projet BD}
\author{BOON Yi Jun \\ DIMASSI Mehdi \\ MOHD NOOR Nur Hakimi \\ NAIR Sritesh \\ TRAORÉ Fiè Victor \date{}}

\begin{document}
\maketitle


\section{Analyse}

Nous avons commencé d'abord par l'analyse du sujet afin d'identifier les données élémentaires, les dépendances fonctionnelles, les contraintes contextuelles et de multiplicité. On a modélisé l'analyse par le tableau ci-dessous:


\begin{table}[h!]
    \centering
    \resizebox{\textwidth}{!}{ 
    \begin{tabular}{|m{5cm}|m{2cm}|m{6cm}|}
        \hline
        \textbf{DF} & \textbf{Contraintes valeur} & \textbf{Contraintes de multiplicité} \\ \hline
        utilisateur $\rightarrow$ produit & quantité $\geq$ 0 & Un utilisateur peut avoir plusieurs produits \\ \hline
        Catégorie,TypeVente $\rightarrow$ salleVente & prixOffre $\geq$ 0 & Une salle de vente contient plusieurs ventes \\ \hline
        idVente $\rightarrow$ produit,salleVente & prixRevient $\geq$ 0 & Un produit a plusieurs caractéristiques \\ \hline
        nomCatégorie $\rightarrow$ descriptionCatégorie & stockVente $\geq$ 0 & Une vente a un seul produit \\ \hline
        idProduit $\rightarrow$ nomProduit,prixRevient,stockVente & prixDépart $>$ 0 & Une vente est dans une seule salle \\ \hline
        idProduit $\rightarrow$ nomCaractéristique &  & Une salle de vente a une seule catégorie \\ \hline
        nomCaractéristique $\rightarrow$ valeurCaractéristique &  & Une offre a un seul utilisateur \\ \hline
        idVente $\rightarrow$ prixDépart,offreMultiple, durée,révocable &  & Un utilisateur peut avoir plusieurs offres \\ \hline
        utilisateur,dateOffre,heureOffre $\rightarrow$ prixOffre,quantité &  &  \\ \hline
        email $\rightarrow$ nom, prénom, adresse &  & \\ \hline
    \end{tabular}}
    \caption{Tableau analyse des dépendances fonctionnelles}
\end{table}

\hspace{10mm}

Les contraintes contextuelles sont :
\begin{itemize}
    \item Accepter l'offre si le prix n'est pas inférieur au prix d'enchère.
    \item Toutes les ventes dans une salle sont du même type.
    \item Si la vente est en offre unique, respecter la contrainte.
    \item La quantité demandé est inférieur au stock.
\end{itemize}

Nous avons ensuite identifiés les entités de la base de données et traduit les DF en schéma entité association suivant :
\begin{figure}[h!]
    \centering
    \includegraphics[width=1\linewidth]{EA.PNG}
    \caption{Schéma entité association}
    \label{fig:enter-label}
\end{figure}


\end{document}